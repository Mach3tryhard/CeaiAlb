\documentclass[12pt]{article}

\usepackage[utf8]{inputenc}
\usepackage{amsmath}
\usepackage{graphicx}
\usepackage{titlesec}
\usepackage{tcolorbox}
\usepackage{titling}
\usepackage{circuitikz}

\title{\bfseries Laborator Electricitate 3}
\author{Sîrghe Matei}
\date{\today}

\pretitle{\begin{center}\LARGE\bfseries}
\posttitle{\par\end{center}\vskip 0.5em}
\preauthor{\begin{center}\large}
\postauthor{\end{center}}
\predate{\begin{center}\large}
\postdate{\end{center}}

\titleformat{\section}
  {\normalfont\Large\bfseries}{\thesection}{1em}{}

\begin{document}

\maketitle

\begin{center}
    \LARGE\textit{Studiul condensatorului electric cu fețe plan-paralele și}
    \LARGE\textit{Determinarea constantei dielectrice a unui izolator}
\end{center}

\section{Teoria Lucrării}

\subsection{Schema Electrică}

\begin{minipage}{0.45\textwidth}
    \centering
    \textbf{Notație condensator :}
    \begin{circuitikz} \draw
        (0,0) to[C] (2,0);
    \end{circuitikz}
\end{minipage}
\hfill
\begin{minipage}{0.45\textwidth}
    \centering
    \begin{circuitikz} \draw
        (0,0) to[battery] (2,0)
        to[C] (4,0)
        to[ground] (5,0);
        \draw (2,1) node[right] {$+Q$};
        \draw (3,1) node[right] {$-Q$};
        \draw (2,-1) node[right] {$5C$};
        \draw (3,-1) node[right] {$-5C$};
    \end{circuitikz}
\end{minipage}


\noindent
\textbf{Definiție : }Condensatorul electric este un dispozitiv format din doua plăci metalice așezate față în față separate de un mediu izolator sau dielectric.
Plăcile metalice ale condensatorului se numesc armături și se încarcă cu aceeași cantitate de sarcină electrică dar de semn opus. Proprietatea
fundamentală a unui condensator electric este aceea de a înmagazina $($reține$)$ pentru un anumit interval de timp sarcină electrică.

\noindent
\textbf{Mărimi fizice :} $[C]_{SI}$ Mărimea fizică care descrie comportamentele unui condensator electric se numește
capacitate electrică și în sistemul internațional de unități se măsoară în \textbf{Farad}.

\begin{center}
    \centering
    \begin{circuitikz} \draw
        (0,0) to[battery] (2,0)
        to[C] (4,0)
        to[ground] (5,0);
        \draw (2,1) node[right] {$+Q$};
        \draw (3,1) node[right] {$-Q$};
        \draw (2,-1) node[right] {$\rho_{N}$};
        \draw (3,-1) node[right] {$\rho_{M}=0$};
    \end{circuitikz}
\end{center}

\noindent
\textbf{Formule fizice :}
\begin{enumerate}
    \item $[C]_{SI}$ = 1 F (Farad) = 1 C/V (Coulomb pe Volt)
    \item $C=\frac{+Q}{\rho_{N}-\rho_{M}}=\frac{-Q}{\rho_{M}-\rho_{N}}=\frac{\left| Q \right|}{U_{NM}}$
    \item $\Delta\rho = U $ - tensiune electrică
    \item $\left| \rho_{M}-\rho_{N} \right| = U_{NM} $ - Diferența de potențial 
\end{enumerate}

\subsection{Montajul Experimental}

\begin{center}
    \centering
    \begin{circuitikz} \draw
        (0,0) to[battery] (2,0)
        to[C] (4,0)
        to[ground] (4,-2)
        to[ground] (5,-2)
        to[ground] (5,0-4)
        (5,0) to[C] (6,0)
        
        (8,0) node[op amp] (opamp) {};
        \draw (opamp.+) -- ++(0,-0.5) node[ground] {};
        \draw (opamp.-) -- ++(0,-0.5) -| (0,0);
        \draw (opamp.out) -- ++(1,0) node[right] {Out};

        to[voltmeter] (8,0)

        \draw (0,-1) node[right] {$[\rho]=V$};
        \draw (0,1) node[right] {$+$};
        \draw (2,1) node[right] {$+Q$};
        \draw (3,1) node[right] {$-Q$};
        \draw (2,-1) node[right] {$5C$};
        \draw (3,-1) node[right] {$-5C$};
    \end{circuitikz}
\end{center}

\section{Datele Experimentului Primare}
Datele experimentale primare reprezintă datele culese din laborator în timpul efectuării lucrării de laborator și trebuie scrise complet așa
cum sunt citite de pe aparatele de măsură. Mărimea fizică citită va fi însoțită de unitatea de măsură aferentă.

\begin{tcolorbox}[colback=yellow!10!white, colframe=black, title=Observație]
De preferat este ca aceste date experimentale să fie înscrise sub forma unor tabele pentru o mai bună organizare.
\end{tcolorbox}

\begin{tcolorbox}[colback=yellow!10!white, colframe=black, title=Observație]
La plecarea din laborator, datele experimentale pentru lucrarea respectivă trebuie să fie complete.
\end{tcolorbox}

\section{Prelucrarea datelor experimentale}
Această secțiune reprezintă aportul avut de fiecare în realizarea referatului de laborator. Dacă datele experimentale primare nu sunt exprimate
în unități ale sistemului internațional de unități, atunci se va face conversia.

\begin{tcolorbox}[colback=yellow!10!white, colframe=black, title=Exemplu]
Din miliamperi în amperi: $1 mA = 10^{-3} A$.
\end{tcolorbox}

Graficele se vor insera la lucrarea de laborator căreia îi aparțin și nu la finalul tuturor lucrărilor de laborator. Graficele se vor face fie folosind
un program specializat, fie pe hârtie milimetrică. Curbele Graficului se vor trasa cu creionul.

\begin{tcolorbox}[colback=yellow!10!white, colframe=black, title=Observație]
Nu sunt acceptate graficele făcute pe hârtia de caiet sau pe foi albe.
\end{tcolorbox}

Numărul zecimalelor indicate în urma calculelor matematice trebuie corelate cu numărul zecimalelor citite în laborator.

\begin{tcolorbox}[colback=yellow!10!white, colframe=black, title=Exemplu]
Am citit 9.4 unități, iar rezultatul este 1.234 unități dar trecem 1.2 deoarece nu putem garanta că 0.034 este precis.
\end{tcolorbox}

\section{Concluzii}
Concluzia referatului trebuie să fie scurtă,(1-2 propoziții) și să indice principalul rezultat obținut după efectuarea lucrării de laborator.

\begin{tcolorbox}[colback=yellow!10!white, colframe=black, title=Exemplu]
Am măsurat puterea de 120 W a unui bec.
\end{tcolorbox}

\begin{tcolorbox}[colback=yellow!10!white, colframe=black, title=Observație]
Referatele de laborator se pot scrie folosind programe de calculator pentru tipărirea lor sau pot fi scrise de mână.
Se poate aduce caiet sau coli albe. Dacă se folosește un caiet, referatele lucrărilor de laborator trebuie să ocupe o parte dedicată a acestuia
și să fie consecutive.
\end{tcolorbox}

\begin{tcolorbox}[colback=yellow!10!white, colframe=black, title=Observație]
Referatul de laborator este individual și se lucrează individual!
\end{tcolorbox}

\end{document}