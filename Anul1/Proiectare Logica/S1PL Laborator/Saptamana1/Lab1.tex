\documentclass[12pt]{article}

\usepackage[a4paper,top=1in,bottom=1in,left=1in,right=1in]{geometry}

\usepackage[utf8]{inputenc}
\usepackage{amsmath}
\usepackage{graphicx}
\usepackage{titlesec}
\usepackage{tcolorbox}
\usepackage{enumitem}
\usepackage{tikz}
\usetikzlibrary{graphs}
\usetikzlibrary{graphdrawing}
\usegdlibrary{force}
\usepackage{amssymb}
\usepackage{hyperref}

\title{\bfseries Laborator Proiectare Logică 1}
\author{Sîrghe Matei}
\date{October 2, 2024}

\titleformat{\section}
  {\normalfont\Large\bfseries}{\thesection}{1em}{}

\begin{document}

\maketitle

\begin{figure}[h!]
    \begin{minipage}{0.6\textwidth}
        Email: \href{mailto:alescandru.chirosca@unibuc.ro}{alescandru.chirosca@unibuc.ro}
        \section{Reprezentarea numerelor}
        \begin{itemize}
            \item Schimbarea de bază de numerație 
            \item Variabile de lungime fixă
            \item Formatul de virgulă mobilă
        \end{itemize}
    \end{minipage}
    \hfill
    \begin{minipage}{0.2\textwidth}
        \begin{tikzpicture}
            \node[circle, draw] (A) at (0,-1.5) {11};
            \node[circle, draw] (B) at (0,0) {10};
            \node[circle, draw] (C) at (0,1.5) {01};
            \node[circle, draw] (D) at (0,3) {00};
            
            \draw[->] (D) -- (C) node[midway, left] {};
            \draw[->] (C) -- (B);
            \draw[->] (B) -- (A);
            \draw[-] (A) -- (1.5,-1.5);
            \draw[-] (1.5,-1.5) -- (1.5,3);
            \draw[->] (1.5,3) -- (D);
        \end{tikzpicture}
    \end{minipage}
\end{figure}

\begin{figure}[h!]
    \begin{minipage}{0.5\textwidth}

        \section{Modelarea proceselor cu inferență directă $f:\mathbb{N}->\mathbb{N}$}
        \begin{itemize}
            \item Porți logice, clasificare
            \item Formele canonice
            \item Optimizări
        \end{itemize}
    \end{minipage}
    \hfill
    \begin{minipage}{0.2\textwidth}
        \begin{tikzpicture}
            \node[circle, draw] (A) at (0,0) {11};
            \node[circle, draw] (B) at (-1,-1) {10};
            \node[circle, draw] (C) at (1,-1) {01};
            \node[circle, draw] (D) at (-1,1) {00};
            
            \draw[->] (D) -- (A) node[midway, left] {};
            \draw[->] (A) -- (C);
            \draw[->] (C) -- (B);
            \draw[->] (B) -- (A);
        \end{tikzpicture}
    \end{minipage}
\end{figure}

\begin{figure}[h!]
    \begin{minipage}{0.6\textwidth}

        \section{Sistemele secvențiale}
        \begin{itemize}
            \item Semnale de CCA (Carry Chain Adder)
            \item Automate și numărătoare
        \end{itemize}
    \end{minipage}
    \hfill
    \begin{minipage}{0.3\textwidth}
        \begin{tikzpicture}
            \node[circle, draw] (A) at (0,2) {00};
            \node[circle, draw] (B) at (-2,0) {11};
            \node[circle, draw] (C) at (0,-2) {10};
            \node[circle, draw] (D) at (2,0) {01};
            
            \draw[->] (A) -- (B) node[midway, right] {x=1};
            \draw[->] (C) -- (B) node[midway, right] {x=1};
            \draw[->] (A) -- (D) node[midway, left] {x=1};
            \draw[->] (C) -- (D) node[midway, left] {x=1};

            \draw[-] (B) -- (-2,2);
            \draw[->] (-2,2) -- (A) node[midway,below] {x=0};
            \draw[-] (B) -- (-2,-2) ;
            \draw[->] (-2,-2) -- (C)node[midway,above] {x=0};

            \draw[-] (D) -- (2,2) ;
            \draw[->] (2,2) -- (A) node[midway,below] {x=0};
            \draw[-] (D) -- (2,-2) ;
            \draw[->] (2,-2) -- (C)node[midway,above] {x=0};
        \end{tikzpicture}
    \end{minipage}
\end{figure}

\end{document}