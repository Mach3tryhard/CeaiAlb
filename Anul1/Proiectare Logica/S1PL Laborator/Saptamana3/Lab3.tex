\documentclass[12pt]{article}

\usepackage[a4paper,top=1in,bottom=1in,left=1in,right=1in]{geometry}

\usepackage[utf8]{inputenc}
\usepackage{float}
\usepackage{amsmath}
\usepackage{graphicx}
\usepackage{titlesec}
\usepackage{tcolorbox}
\usepackage{enumitem}
\usepackage{tikz}
\usetikzlibrary{graphs}
\usetikzlibrary{graphdrawing}
\usegdlibrary{force}
\usepackage{amssymb}
\usepackage{hyperref}

\title{\bfseries Laborator Proiectare Logică 3}
\author{Sîrghe Matei}
\date{October 16, 2024}

\titleformat{\section}
  {\normalfont\Large\bfseries}{\thesection}{1em}{}
\begin{document}
\maketitle

\begin{center}
    \large \textbf{Reprezentarea numerelor}
\end{center}

\begin{enumerate}[label=(\alph*)]
    \item { \textbf{Numerele întregi $ y \in \mathbb{Z} $}
        \begin{itemize}
            \item De regulă se reprezintă pe 32 de biți.
            \item { Fără semn $ x \in \mathbb{Z^{+}}+{\varnothing}$ le reprezentăm direct.\\
            $ 124_{(10)}=2^{6}+2^{5}+2^{4}+2^{1}=...00001110010 $\\
            val max $= FFFFFFFF=2^{33}-1 = \sum_{i=0}^{32}2^{i}=8589934591$\\
            val min $= \varnothing $
            }
            \item {Cu semn $ x \in \mathbb{Z}$\\
            $(\forall)$ k, lățimea reg.\\
            Valoarea stocată = offset $+ v_{\uparrow}=V_{s}$\\
            offset = $s^{k-1}-1$\\
            \textbf{EX:} $ v_{\uparrow} = -247 $\\
            $ V_{s}=2^{32-1}-1+(-247) = 2^{31}-1-247 =2147483647-247=214748400 $\\
            $ = 2^30+2^29+...+2^8+2^6+2^3=0111111111111111111111101001000$\\
            Biții înafară de numărul propriu zis sunt inversați la numere negative.\\
            $100000000000000000000000011110110$\\
            $=2^31+3^7+2^6+2^5+2^4+2^2+2^1$\\
            
            \begin{figure}[h!]
                \begin{minipage}{0.5\textwidth}
                    \hspace{2cm} $ v_{\uparrow}=247$
                \end{minipage}
                \hfill
                \begin{minipage}{0.4\textwidth}
                    \begin{tikzpicture}
                        \node at (0,0.25) {2};
                        \draw[-] (0,-4.5) -- (0,0);
                        
                        \node at (-0.4,-0.25) {247};
                        \node at (0.25,-0.25) {1};
                        \node at (-0.4,-0.75) {123};
                        \node at (0.25,-0.75) {1};
                        \node at (-0.25,-1.25) {61};
                        \node at (0.25,-1.25) {1};
                        \node at (-0.25,-1.75) {30};
                        \node at (0.25,-1.75) {0};
                        \node at (-0.25,-2.25) {15};
                        \node at (0.25,-2.25) {1};
                        \node at (-0.25,-2.75) {7};
                        \node at (0.25,-2.75) {1};
                        \node at (-0.25,-3.25) {3};
                        \node at (0.25,-3.25) {1};
                        \node at (-0.25,-3.75) {1};
                        \node at (0.25,-3.75) {1};
                        \node at (-0.25,-4.25) {0};
                    
                        \draw[->] (1,-4) -- (1,0) node[midway, right] {MSB};
                    \end{tikzpicture}
                \end{minipage}
            \end{figure}

            }
        \end{itemize}
    }
    \item { \textbf{Numere reale $ x \in \mathbb{R} $}\\
        De regulă se reprezintă pe 64 de biți. (k=11)\\
        $ 0,00031 = 3,14 \times 10^{-4} = 3,14E-4 $\\
        $ 247,32 = 2,473 \times 10^{2} = 2,473E+2$\\
        10 - baza noastră\\
        $ 0,0247_{10}=0,0000011001_{2} $
    }
\end{enumerate}

\begin{figure}[H]
    \begin{minipage}{0.25\textwidth}
        \vspace{0.6cm}
        $0,0247\times 2 = 0,0494$\\
        $0,0494\times 2 = 0,0988$\\
        $0,0988\times 2 = 0,1976$\\
        $0,1976\times 2 = 0,3952$\\
        $0,3952\times 2 = 0,7904$\\
        $0,7904\times 2 = 1,5808$\\
        $0,5808\times 2 = 1,1616$\\
        .\hspace{3cm}.\\
        .\hspace{3cm}.\\
        .\hspace{3cm}.\\
    \end{minipage}
    \hspace{0.2cm}
    \begin{minipage}{0.2\textwidth}
        \begin{tikzpicture}
            \draw[->] (0,0) -- (0,-5) node[midway, right] {LSB};
            \node at (-0.25,-0.25) {0};
            \node at (-0.25,-0.75) {0};
            \node at (-0.25,-1.25) {0};
            \node at (-0.25,-1.75) {0};
            \node at (-0.25,-2.25) {0};
            \node at (-0.25,-2.75) {1};
            \node at (-0.25,-3.25) {1};
            \node at (-0.25,-3.75) {0};
            \node at (-0.25,-4.25) {0};
            \node at (-0.25,-4.75) {1};
        \end{tikzpicture}
    \end{minipage}
    \hfill
    \begin{minipage}{0.5\textwidth}
        \vspace{-0.5cm}
        exp = offset + $e_{\uparrow}$ \\
        exp = $ 2^{k-1}-1+e_{\uparrow} $\\
        $e_{\uparrow}=-6 $\\
        $k=11$ \\
        offset $= 2^{k-1}-1=2^{10}-1+(-6) $\\
        $= 1024 -1-6=1023-6=1017 $\\
    \end{minipage}
\end{figure}

\end{document}