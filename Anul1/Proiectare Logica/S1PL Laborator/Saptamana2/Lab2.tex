\documentclass[12pt]{article}

\usepackage[a4paper,top=1in,bottom=1in,left=1in,right=1in]{geometry}

\usepackage[utf8]{inputenc}
\usepackage{amsmath}
\usepackage{graphicx}
\usepackage{titlesec}
\usepackage{tcolorbox}
\usepackage{enumitem}
\usepackage{tikz}
\usetikzlibrary{graphs}
\usetikzlibrary{graphdrawing}
\usegdlibrary{force}
\usepackage{amssymb}
\usepackage{hyperref}

\title{\bfseries Laborator Proiectare Logică 2}
\author{Sîrghe Matei}
\date{October 9, 2024}

\titleformat{\section}
  {\normalfont\Large\bfseries}{\thesection}{1em}{}
\begin{document}
\maketitle

\begin{center}
    \large \textbf{Exerciții schimbare bază de numerație}
\end{center}

\begin{equation}
    (\forall) x \in \mathbb{R} , x= \sum_{i(\forall)} a_{i} b^{i} ; a \in \mathbb{Z}, b \in \mathbb{Z^{+}}, a<b, b<1
\end{equation}

\begin{align*}
    & 10110_{(2)} = 112_{(4)} = 26_{(8)} = 16_{(16)} = 2^{1} + 2^{2} + 2^{4}= 22_{(10)}\\
    & 10010101_{(2)} = 2111_{(4)} = 225_{(8)} = 95_{(16)} = 2^{0}+2^{2}+2^{4}+2^{7} = 149_{(10)}\\
    & 100100001001_{(2)}= 21021_{(4)} =4411_{(8)}=909_{(16)}=2^{0}+ 2^{3}+ 2^{8}+ 2^{11}=2313_{(10)}\\
    & 10010_{(2)}=102_{(4)}=22_{(8)}=12_{(16)}=2^{1}+ 2^{4}=18_{(10)}\\
    & 111011_{(2)}=323_{(4)}=73_{(8)}=3B_{(16)}=2^{0}+2^{1}+2^{3}+2^{4}+2^{5}=59_{(10)}\\
    & 11111111_{(2)}=3333_{(4)}=377_{(8)}=FF_{(16)}=2^{0}+2^{1}+2^{2}+2^{3}+2^{4}+2^{5}+2^{6}+2^{7}=255_{(10)}\\
\end{align*}

\begin{figure}[h!]
    \begin{minipage}{0.3\textwidth}
        $27_{(10)}=33_{(8)}$
    \end{minipage}
    \hfill
    \begin{minipage}{0.3\textwidth}
        $33_{(10)}=100001_{(2)}$
    \end{minipage}
    \hfill
    \begin{minipage}{0.3\textwidth}
        $1859_{(10)}=3503_{(8)}$
    \end{minipage}
\end{figure}

\begin{figure}[h!]
    \begin{minipage}{0.3\textwidth}
        \begin{tikzpicture}
            \node at (0,0.25) {8};
            \draw[-] (0,-1.5) -- (0,0);
            
            \node at (-0.25,-0.25) {27};
            \node at (0.25,-0.25) {3};
            \node at (-0.25,-0.75) {3};
            \node at (0.25,-0.75) {3};
            \node at (-0.25,-1.25) {0};
        
            \draw[->] (1,-1) -- (1,0) node[midway, right] {MSB};
        \end{tikzpicture}
    \end{minipage}
    \hfill
    \begin{minipage}{0.3\textwidth}
        \begin{tikzpicture}
            \node at (0,0.25) {2};
            \draw[-] (0,-3.5) -- (0,0);
            
            \node at (-0.25,-0.25) {33};
            \node at (0.25,-0.25) {1};
            \node at (-0.25,-0.75) {16};
            \node at (0.25,-0.75) {0};
            \node at (-0.25,-1.25) {8};
            \node at (0.25,-1.25) {0};
            \node at (-0.25,-1.75) {4};
            \node at (0.25,-1.75) {0};
            \node at (-0.25,-2.25) {2};
            \node at (0.25,-2.25) {0};
            \node at (-0.25,-2.75) {1};
            \node at (0.25,-2.75) {1};
            \node at (-0.25,-3.25) {0};

            \draw[->] (1,-3) -- (1,0) node[midway, right] {MSB};
        \end{tikzpicture}
    \end{minipage}
    \hfill
    \begin{minipage}{0.3\textwidth}
        \begin{tikzpicture}
            \node at (0,0.25) {2};
            \draw[-] (0,-2.5) -- (0,0);
            
            \node at (-0.5,-0.25) {1859};
            \node at (0.25,-0.25) {3};
            \node at (-0.4,-0.75) {232};
            \node at (0.25,-0.75) {0};
            \node at (-0.25,-1.25) {29};
            \node at (0.25,-1.25) {5};
            \node at (-0.25,-1.75) {3};
            \node at (0.25,-1.75) {3};
            \node at (-0.25,-2.25) {0};

            \draw[->] (1,-2) -- (1,0) node[midway, right] {MSB};
        \end{tikzpicture}
    \end{minipage}
\end{figure}

\textbf{Problemă :}
Câte numere există între $175_{(8)}$ și $200_{(8)}$.\\
Există 2 numere : $175_{(8)}, 176_{(8)}, 177_{(8)}, 200_{(8)}$.

\begin{equation}
    n_{(10)}={b_{n}b_{n-1}...b_{1}b_{0}}_{(2)}={\sum_{i=0}^{n}b_{i}a^{i}}_{(a)}
\end{equation}

\begin{align*}
    & 6D_{(16)}=1101101_{(2)}=1231_{(4)}=155_{(8)}=109_{(10)}\\
    & 743_{(16)}=11101000011_{(2)}=131003_{(4)}=3503_{(8)}=7\times 16^2 + 4 \times 16^1 +3 \times 16^0 = 1859_{(10)}\\
    & 37FD_{(16)}=11011111111101_{(2)}=11211_{(4)}=545_{(8)}=14333_{(10)}\\
    & 165_{(16)}=101100101_{(2)}=11211_{(4)}=545_{(8)}=357_{(10)}\\
    & ABCD_{(16)}=1010101111001101_{(2)}=22233031_{(4)}=125715_{(8)}=43981_{(10)}\\
    & 7FF_{(16)}=11111111111_{(2)}=133333_{(4)}=3777_{(8)}=2047_{(10)}\\
    & E71_{(16)}=111001110001_{(2)}=321302_{(4)}=7161_{(8)}=3697_{(10)}\\
\end{align*}

\begin{align*}
    & 0,0000011001=2^{-6}+2^{-7}+2^{-10}=2^{-6}(1+2^{-1}+2^{4})\\
    & =2^{-6}(1+2^{-1}+2^{-4}) \text{primul bit se ignoră este standard}\\
    & 0|0111111001|10010......| \\
    & e_{\uparrow(10)}=\frac{[log_{10}|10|]}{log_{10}2}_{\uparrow} \\
    & m_{(10)}=\frac{|10|}{2^{e_{\uparrow}}}-1\\
    & D4EA,71_{(16)}=110101001110101001110001 \\
    & = 2^{15}+2^{14}+2^{12}+2^{10}+2^7+2^6+2^5+2^3+2^1+2^{-2}+2^{-3}+2^{-4}+2^{-8} \\
    & = 2^{15}(1+2^{-1}+2^{-3}+2^{-5}+2^{-8}+2^{-9}+2^{-10}+2^{-12}+2^{-14}+2^{-17}+2^{-18}+2^{-19}+2^{-23}) \\
    & e_{\uparrow}=15 \\
    & k=11 \\
    & exp=1023+15=2^10+2^3+2^2+2^1 \\
    & 0|10000001110|10101001110101001110......| \\
\end{align*}

\end{document}