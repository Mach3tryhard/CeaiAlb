\documentclass[12pt]{article}

\usepackage[utf8]{inputenc}
\usepackage{amsmath}
\usepackage{graphicx}
\usepackage{titlesec}
\usepackage{tcolorbox}

\title{\bfseries Laborator Electricitate 1}
\author{Sîrghe Matei}
\date{\today}

\titleformat{\section}
  {\normalfont\Large\bfseries}{\thesection}{1em}{}

\begin{document}

\maketitle

\section{Teoria Lucrării}
În cadrul acestei secțiuni, vor fi discutate procedeele și fenomenele fizice care vor fi utilizate în cadrul lucrării de laborator împreună
cu mărimile fizice aferente și unitățile lor de măsură. Notițele scrise în sala de seminar pot fi folosite integral pentru această secțiune.

\section{Datele Experimentului Primare}
Datele experimentale primare reprezintă datele culese din laborator în timpul efectuării lucrării de laborator și trebuie scrise complet așa
cum sunt citite de pe aparatele de măsură. Mărimea fizică citită va fi însoțită de unitatea de măsură aferentă.

\begin{tcolorbox}[colback=yellow!10!white, colframe=black, title=Observație]
De preferat este ca aceste date experimentale să fie înscrise sub forma unor tabele pentru o mai bună organizare.
\end{tcolorbox}

\begin{tcolorbox}[colback=yellow!10!white, colframe=black, title=Observație]
La plecarea din laborator, datele experimentale pentru lucrarea respectivă trebuie să fie complete.
\end{tcolorbox}

\section{Prelucrarea datelor experimentale}
Această secțiune reprezintă aportul avut de fiecare în realizarea referatului de laborator. Dacă datele experimentale primare nu sunt exprimate
în unități ale sistemului internațional de unități, atunci se va face conversia.

\begin{tcolorbox}[colback=yellow!10!white, colframe=black, title=Exemplu]
Din miliamperi în amperi: $1 mA = 10^{-3} A$.
\end{tcolorbox}

Graficele se vor insera la lucrarea de laborator căreia îi aparțin și nu la finalul tuturor lucrărilor de laborator. Graficele se vor face fie folosind
un program specializat, fie pe hârtie milimetrică. Curbele Graficului se vor trasa cu creionul.

\begin{tcolorbox}[colback=yellow!10!white, colframe=black, title=Observație]
Nu sunt acceptate graficele făcute pe hârtia de caiet sau pe foi albe.
\end{tcolorbox}

Numărul zecimalelor indicate în urma calculelor matematice trebuie corelate cu numărul zecimalelor citite în laborator.

\begin{tcolorbox}[colback=yellow!10!white, colframe=black, title=Exemplu]
Am citit 9.4 unități, iar rezultatul este 1.234 unități dar trecem 1.2 deoarece nu putem garanta că 0.034 este precis.
\end{tcolorbox}

\section{Concluzii}
Concluzia este reprezentată de o sinteză a rezultatelor obținute în cadrul lucrării de laborator.
Aceasta trebuie să fie scrisă într-un mod clar și concis, maxim 3 fraze.

\end{document}